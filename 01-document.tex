%------------------------------------------------------------------------------------------------------------------------------%
%                                                          Title Page                                                          %
%------------------------------------------------------------------------------------------------------------------------------%

\thispagestyle{empty} % Removes page numbering from the first page
\flushbottom % Makes all text pages the same height
\maketitle % Print the title and abstract box


%------------------------------------------------------------------------------------------------------------------------------%
%                                                           License                                                            %
%------------------------------------------------------------------------------------------------------------------------------%

\section*{License}

    \scriptsize\noindent%
    \begin{minipage}{\columnwidth}
        \centering\tt
        \includegraphics[height=6.0mm]{cc/by.pdf}\\[0.5\smallskipamount]
        {\scriptsize\url{https://creativecommons.org/licenses/by/4.0/}}
    \end{minipage}
    \normalsize


%------------------------------------------------------------------------------------------------------------------------------%
%                                                      Table Of Contents                                                       %
%------------------------------------------------------------------------------------------------------------------------------%

\tableofcontents


%------------------------------------------------------------------------------------------------------------------------------%
%                                                         Nomenclature                                                         %
%------------------------------------------------------------------------------------------------------------------------------%

\section*{Nomenclature}
\addcontentsline{toc}{section}{Nomenclature}

\newlength{\lencsep}\setlength{\lencsep}{0.8em}
\newlength{\lensymb}\setlength{\lensymb}{2.5em}
\newlength{\lendefn}\setlength{\lendefn}{4.0em}
\newlength{\lenwhat}\setlength{\lenwhat}{\linewidth}
\newlength{\lenWHAT}\setlength{\lenWHAT}{\linewidth}
\addtolength{\lenwhat}{-\lensymb}
\addtolength{\lenwhat}{-\lendefn}
\addtolength{\lenwhat}{-\lencsep}
\addtolength{\lenWHAT}{-\lensymb}
\par\noindent\begin{supertabular}{@{}p{\lensymb}@{}p{\lenwhat}@{\hspace{\lencsep}}p{\lendefn}}
    \multicolumn{3}{@{}l}{\em Latin symbols:} \\
    $c$             & Specific heat, \kilo\joule\per\kilogram\usk\kelvin,           & Eq.~(\ref{eq:def.gamma})          \\
    $D$             & Piston diameter, \meter,                                      & Eq.~(\ref{eq:S})                  \\
    $K$             & Heat-to-work transfer ratio, ---,                             & Eq.~(\ref{eq:def.K})              \\
    $L$             & Connecting rod length, \meter,                                & Eq.~(\ref{eq:x})                  \\
    $n$             & Polytropic exponent, ---,                                     & Eq.~(\ref{eq:n})                  \\
    $R$             & Crank radius, \meter,                                         & Eq.~(\ref{eq:S})                  \\
    $r$             & Compression ratio, ---,                                       & Eq.~(\ref{eq:r})                  \\
    $S$             & Stroke, \meter,                                               & Eq.~(\ref{eq:S})                  \\
    $V$             & Volume, \meter\cubed,                                         & Eq.~(\ref{eq:Vd})                 \\
    $x$             & Piston position from the TDS, \meter,                         & Eq.~(\ref{eq:x})                  \\
    $z$             & Engine's cylinder count, ---,                                 & Eq.~(\ref{eq:Vd})                 \\[6pt]
    \multicolumn{3}{@{}l}{\em Greek symbols:} \\
    $\gamma$        & Specific heat ratio, ---,                                     & Eq.~(\ref{eq:def.gamma})          \\[6pt]
    \multicolumn{3}{@{}l}{\em Subscripts:} \\
    $d$             & \multicolumn{2}{@{}p{\lenWHAT}}{displaced}                                                        \\
    $i$             & \multicolumn{2}{@{}p{\lenWHAT}}{process index}                                                    \\
    $LR$            & \multicolumn{2}{@{}p{\lenWHAT}}{of $L/R$}                                                         \\
    $max$           & \multicolumn{2}{@{}p{\lenWHAT}}{maximum}                                                          \\
    $min$           & \multicolumn{2}{@{}p{\lenWHAT}}{minimum}                                                          \\
    $P$             & \multicolumn{2}{@{}p{\lenWHAT}}{taken at constant pressure}                                       \\
    $SD$            & \multicolumn{2}{@{}p{\lenWHAT}}{of $S/D$}                                                         \\
    $u$             & \multicolumn{2}{@{}p{\lenWHAT}}{unit or unitary}                                                  \\
    $v$             & \multicolumn{2}{@{}p{\lenWHAT}}{taken at constant specific volume}                                \\[6pt]
    \multicolumn{3}{@{}l}{\em Superscripts:} \\
    $j$             & \multicolumn{2}{@{}p{\lenWHAT}}{polytropic exponent loop index}                                   \\
\end{supertabular}


%------------------------------------------------------------------------------------------------------------------------------%
%                                                         Introduction                                                         %
%------------------------------------------------------------------------------------------------------------------------------%

\section{Introduction}

    The air-standard, finite-time heat addition Otto engine model~\cite{2017-NaaktgeborenC-IntJMechEngEduc}, or `FTHA Otto'  for
    short, is a pure-substance, equilibrium engineering thermodynamics teaching model aimed at senior  undergraduate  Mechanical
    Engineering (ME) students.

    The model captures the effects of non-in\-stan\-ta\-neous  heat  addition  in  Otto  engine  model  types  by  allowing  for
    simultaneous heat and work interactions of the working closed system. It  uses  process  discretization  into  subprocesses,
    which are homogeneously modeled as local polytropic processes.

    The solution procedure for the FTHA is the determination of (i)~initial state and (ii)~polytropic exponent for each  one  of
    the subprocesses along the cycle. Initial state determination requires a marching---in time, or, equivalently, engine  crank
    angle---solution strategy, in which the next subprocess initial state coincides with the  current  process  end  state.  The
    determination of the polytropic exponent, on the other hand, requires an iterative prediction-correction procedure; hence, a
    nested, multi-step inner loop that may require yet another nested inner loop for state determination  whenever  the  working
    substance specific heat is a function of the temperature, which is the most common case.

    A deeper look into polytropic processes~\cite{2012-ChristiansJ-IntJMechEngEduc, 2020-NaaktgeborenC-Polytropic-engrXiv-rev02}
    reveals that the polytropic exponent can be \emph{directly} determined  for  a  local  polytropic  process  if  the  working
    substance is an ideal gas~\cite{2020-NaaktgeborenC-Polytropic-engrXiv-rev02} and if the heat-to-work transfer ratio%
    %
    \begin{equation}
        K \equiv \frac{\delta q}{\delta w},
        \label{eq:def.K}
    \end{equation}
    %
    \noindent also referred to as  energy  transfer  ratio~\cite{2012-ChristiansJ-IntJMechEngEduc},  where  $\delta  q$  is  the
    differential heat transfer to the system, and $\delta w$ is the differential work transfer from  the  system;  and  specific
    heat ratio%
    %
    \begin{equation}
        \gamma \equiv \frac{c_P}{c_v},
        \label{eq:def.gamma}
    \end{equation}
    %
    \noindent where $c_P$ and $c_v$ are the isobaric and isochoric specific heats, are known---the resulting polytropic exponent
    is   given   on   references~\cite{2012-ChristiansJ-IntJMechEngEduc,   2020-NaaktgeborenC-Polytropic-engrXiv-rev02} as%
    %
    \begin{equation}
        n = \gamma + K(1 - \gamma),
        \label{eq:n}
    \end{equation}
    %
    \noindent while theoretical requisites are further discussed in~\cite{2020-NaaktgeborenC-Polytropic-engrXiv-rev02}.

    As FTHA Otto can meet the necessary conditions for determining the polytropic exponents base on $K$ and $\gamma$, this means
    that its solution algorithm can be analytically improved as to reduce the number of nested  loops  or  even  eliminate  them
    altogether, which may lead to improved learning and faster computer implementations.

    Therefore, this  work  proposes  an  improved  FTHA  Otto  model  that  incorporates  local  polytropic  process  analytical
    simplifications on the original FTHA Otto model, along with other minor improvements that are highlighted  as  they  appear.
    For didactic reasons, a complete improved model is herein presented, instead of just the differences and improvements.


%------------------------------------------------------------------------------------------------------------------------------%
%                                                   Improved FTHA Otto Model                                                   %
%------------------------------------------------------------------------------------------------------------------------------%

\section{Improved FTHA Otto Model}

    Following air-standard modeling, the thermodynamic system extent is a single closed  compressible  system  representing  the
    inside volume of a single engine's combustion chamber. The operation in cyclical with no mass transfer, and  the  combustion
    and exhaust processes are replaced by heat transfer interactions. Heat addition occurs in a \emph{finite time},  while  heat
    extraction   is   kept   instantaneous.   For   additional   discussion,    please    refer    to    the    original    FTHA
    paper~\cite{2017-NaaktgeborenC-IntJMechEngEduc}.

    Following Brunetti~\cite{2012-BrunettiF-Blucher}, the dynamics of the piston-crank mechanism are accounted  for  as  far  as
    volume variation is concerned. Thus, system minimum (dead), maximum and engine displaced volumes are  $V_{min}$,  $V_{max}$,
    and%
    %
    \begin{equation}
        V_d = z(V_{max} - V_{min}) = zV_{du},
        \label{eq:Vd}
    \end{equation}
    %
    \noindent where $z$ is the engine number of cylinders and $V_{du}$ the unit displacement volume. Let further $S$,  $D$,  and
    $r_{SD}$ be the piston stroke, diameter, and their ratio, respectively. Moreover, let $R$, $L$, and $r_{LR}$  be  the  crank
    radius, conecting rod length, and their ratio, respectively. Let further $x$ and $\alpha$ be the instantaneous piston
    position, measured from the top dead center (TDC), and crank angular position, respectively, so that $\alpha = 0$ at the
    TDC, then, the following hold~\cite{2012-BrunettiF-Blucher, 2017-NaaktgeborenC-IntJMechEngEduc}:%
    %
    \begin{align}
        \label{eq:S}
        S           &= 2R = \frac{4 V_{du}}{\pi D^{2}}, \\
        \label{eq:Vdu}
        V_{du}      &= V_{max} - V_{min} = \frac{\pi S}{4} D^{2}, \\
        \label{eq:r}
        r           &= \frac{V_{max}}{V_{min}} = 1 + \frac{V_{du}}{V_{min}}, \\
        \label{eq:x}
        x(\alpha)   &=
            L\left(
                1 - \sqrt{
                    1 - \left(\frac{R}{L}\sin\alpha\right)^2
                }
            \right) \nonumber\\
                    &+ R(1 - \cos\alpha).
    \end{align}


