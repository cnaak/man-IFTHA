%------------------------------------------------------------------------------------------------------------------------------%
%                                                          Title Page                                                          %
%------------------------------------------------------------------------------------------------------------------------------%

\thispagestyle{empty} % Removes page numbering from the first page
\flushbottom % Makes all text pages the same height
\maketitle % Print the title and abstract box


%------------------------------------------------------------------------------------------------------------------------------%
%                                                           License                                                            %
%------------------------------------------------------------------------------------------------------------------------------%

\section*{License}

    \scriptsize\noindent%
    \begin{minipage}{\columnwidth}
        \centering\tt
        \includegraphics[height=6.0mm]{cc/by.pdf}\\[0.5\smallskipamount]
        {\scriptsize\url{https://creativecommons.org/licenses/by/4.0/}}
    \end{minipage}
    \normalsize


%------------------------------------------------------------------------------------------------------------------------------%
%                                                      Table Of Contents                                                       %
%------------------------------------------------------------------------------------------------------------------------------%

\tableofcontents


%------------------------------------------------------------------------------------------------------------------------------%
%                                                         Nomenclature                                                         %
%------------------------------------------------------------------------------------------------------------------------------%

\section*{Nomenclature}
\addcontentsline{toc}{section}{Nomenclature}

\newlength{\lencsep}\setlength{\lencsep}{0.8em}
\newlength{\lensymb}\setlength{\lensymb}{2.5em}
\newlength{\lendefn}\setlength{\lendefn}{4.0em}
\newlength{\lenwhat}\setlength{\lenwhat}{\linewidth}
\newlength{\lenWHAT}\setlength{\lenWHAT}{\linewidth}
\addtolength{\lenwhat}{-\lensymb}
\addtolength{\lenwhat}{-\lendefn}
\addtolength{\lenwhat}{-\lencsep}
\addtolength{\lenWHAT}{-\lensymb}
\par\noindent\begin{supertabular}{@{}p{\lensymb}@{}p{\lenwhat}@{\hspace{\lencsep}}p{\lendefn}}
    \multicolumn{3}{@{}l}{\em Latin symbols:} \\
    $c$             & Specific heat, \kilo\joule\per\kilogram\usk\kelvin,           & Eq.~(\ref{eq:def.gamma})          \\
    $K$             & Heat-to-work transfer ratio, ---,                             & Eq.~(\ref{eq:def.K})              \\
    $n$             & Polytropic exponent, ---,                                     & Eq.~(\ref{eq:n})                  \\[6pt]
    \multicolumn{3}{@{}l}{\em Greek symbols:} \\
    $\gamma$        & Specific heat ratio, ---,                                     & Eq.~(\ref{eq:def.gamma})          \\[6pt]
    \multicolumn{3}{@{}l}{\em Subscripts:} \\
    $i$             & \multicolumn{2}{@{}p{\lenWHAT}}{process index}                                                    \\
    $P$             & \multicolumn{2}{@{}p{\lenWHAT}}{taken at constant pressure}                                       \\
    $v$             & \multicolumn{2}{@{}p{\lenWHAT}}{taken at constant specific volume}                                \\[6pt]
    \multicolumn{3}{@{}l}{\em Superscripts:} \\
    $j$             & \multicolumn{2}{@{}p{\lenWHAT}}{polytropic exponent loop index}                                   \\
\end{supertabular}


%------------------------------------------------------------------------------------------------------------------------------%
%                                                         Introduction                                                         %
%------------------------------------------------------------------------------------------------------------------------------%

\section{Introduction}

    The air-standard, finite-time heat addition Otto engine model~\cite{2017-NaaktgeborenC-IntJMechEngEduc}, or `FTHA Otto'  for
    short, is a pure-substance, equilibrium engineering thermodynamics teaching model aimed at senior  undergraduate  Mechanical
    Engineering (ME) students.

    The model captures the effects of non-in\-stan\-ta\-neous  heat  addition  in  Otto  engine  model  types  by  allowing  for
    simultaneous heat and work interactions of the working closed system. It  uses  process  discretization  into  subprocesses,
    which are homogeneously modeled as local polytropic processes.

    The solution procedure for the FTHA is the determination of (i)~initial state and (ii)~polytropic exponent for each  one  of
    the subprocesses along the cycle. Initial state determination requires a marching---in time, or, equivalently, engine  crank
    angle---solution strategy, in which the next subprocess initial state coincides with the  current  process  end  state.  The
    determination of the polytropic exponent, on the other hand, requires an iterative prediction-correction procedure; hence, a
    nested, multi-step inner loop that may require yet another nested inner loop for state determination  whenever  the  working
    substance specific heat is a function of the temperature, which is the most common case.

    A deeper look into polytropic processes~\cite{2012-ChristiansJ-IntJMechEngEduc, 2020-NaaktgeborenC-Polytropic-engrXiv-rev02}
    reveals that the polytropic exponent can be \emph{directly} determined  for  a  local  polytropic  process  if  the  working
    substance is an ideal gas~\cite{2020-NaaktgeborenC-Polytropic-engrXiv-rev02} and if the heat-to-work transfer ratio%
    %
    \begin{equation}
        K \equiv \frac{\delta q}{\delta w},
        \label{eq:def.K}
    \end{equation}
    %
    \noindent also referred to as  energy  transfer  ratio~\cite{2012-ChristiansJ-IntJMechEngEduc},  where  $\delta  q$  is  the
    differential heat transfer to the system, and $\delta w$ is the differential work transfer from  the  system;  and  specific
    heat ratio%
    %
    \begin{equation}
        \gamma \equiv \frac{c_P}{c_v},
        \label{eq:def.gamma}
    \end{equation}
    %
    \noindent where $c_P$ and $c_v$ are the isobaric and isochoric specific heats, are known---the resulting polytropic exponent
    is   given   on   references~\cite{2012-ChristiansJ-IntJMechEngEduc,   2020-NaaktgeborenC-Polytropic-engrXiv-rev02} as%
    %
    \begin{equation}
        n = \gamma + K(1 - \gamma),
        \label{eq:n}
    \end{equation}
    %
    \noindent while
    theoretical requisites are further discussed in~\cite{2020-NaaktgeborenC-Polytropic-engrXiv-rev02}.

    As FTHA Otto can meet the necessary conditions for determining the polytropic exponents base on $K$ and $\gamma$, this means
    that its solution algorithm can be analytically improved as to reduce the number of nested  loops  or  even  eliminate  them
    altogether, which may lead to improved learning and faster computer implementations.

    Therefore, this  work  proposes  an  improved  FTHA  Otto  model  that  incorporates  local  polytropic  process  analytical
    simplifications on the original FTHA Otto model, along with other minor improvements that are highlighted as they appear.


%------------------------------------------------------------------------------------------------------------------------------%
%                                                   Improved FTHA Otto Model                                                   %
%------------------------------------------------------------------------------------------------------------------------------%

\section{Improved FTHA Otto Model}



