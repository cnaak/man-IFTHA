%------------------------------------------------------------------------------------------------------------------------------%
%                                                          Title Page                                                          %
%------------------------------------------------------------------------------------------------------------------------------%

\thispagestyle{empty} % Removes page numbering from the first page
\flushbottom % Makes all text pages the same height
\maketitle % Print the title and abstract box


%------------------------------------------------------------------------------------------------------------------------------%
%                                                           License                                                            %
%------------------------------------------------------------------------------------------------------------------------------%

\section*{License}

    \scriptsize\noindent%
    \begin{minipage}{\columnwidth}
        \centering\tt
        \includegraphics[height=6.0mm]{cc/by.pdf}\\[0.5\smallskipamount]
        {\scriptsize\url{https://creativecommons.org/licenses/by/4.0/}}
    \end{minipage}
    \normalsize


%------------------------------------------------------------------------------------------------------------------------------%
%                                                      Table Of Contents                                                       %
%------------------------------------------------------------------------------------------------------------------------------%

\tableofcontents


%------------------------------------------------------------------------------------------------------------------------------%
%                                                         Introduction                                                         %
%------------------------------------------------------------------------------------------------------------------------------%

\section{Introduction}

    The air-standard, finite-time heat addition Otto engine model~\cite{2017-NaaktgeborenC-IntJMechEngEduc}, or `FTHA Otto'  for
    short, is a pure-substance, equilibrium engineering thermodynamics teaching model aimed at senior  undergraduate  Mechanical
    Engineering (ME) students that captures the effects of non-instantaneous  heat  addition  in  Otto  engine  model  types  by
    allowing for simultaneous heat and work interactions of the gaseous closed system formed by the piston-cylinder arrangement.
    It uses process discretization into $i$-subprocesses, along with homogeneous local polytropic modeling, whose  proposal  was
    further developed on reference~\cite{2020-NaaktgeborenC-Polytropic-engrXiv-rev02}.

    Since all FTHA Otto model processes are modeled as local polytropic ones, the solution procedure boils down to  finding  the
    polytropic exponent $n_i$, for each $i$-th process, that satisfies the first law of thermodynamics for the  given  substance
    model---usually an ideal gas one---and external (engine) volume constraints.

    The solution procedure proposed in the original FTHA Otto cycle involves an iterative process for each $n_i$ exponent,  that
    yields successive corrections $n_i^j$ for an initial isentropic guess $n_i^0$, that converge  towards  the  $n_i$  solution.
    This means the solution for the \emph{cycle} involves \emph{two nested loops}: an outer `$i$-loop'  of  sub-processes  along
    the cycle, each having an inner `$j$-loop' for the polytropic exponent convergence.

    Original FTHA's $j$-loops entail (i)~solving  an  energy  balance  with  the  previously-guessed  $n_i^j$  for  the  working
    substance's specific internal energy, (ii)~solving for the temperature (iii)~and pressure with the substance's specific heat
    and state models, which may incur in additional iterative procedures, before finally (iv)~yielding a corrected $n_i^{j+1}$.

    A deeper look into polytropic processes~\cite{2012-ChristiansJ-IntJMechEngEduc, 2020-NaaktgeborenC-Polytropic-engrXiv-rev02}
    reveals that the polytropic exponent can be \emph{directly} determined  for  a  local  polytropic  process  if  the  working
    substance is an ideal gas~\cite{2020-NaaktgeborenC-Polytropic-engrXiv-rev02} and if the heat-to-work  transfer  ratio,  also
    referred to as energy transfer ratio~\cite{2012-ChristiansJ-IntJMechEngEduc}, and specific heat ratio are known.

    Since these conditions are met for FTHA Otto cycles with ideal gas working substance---the usual application---it means  the
    FTHA  Otto  solution  procedure  can  be  analytically  simplified,  leading  to  improved  learning  and  faster   computer
    implementations.

    Therefore, this  work  proposes  an  improved  FTHA  Otto  model  that  incorporates  local  polytropic  process  analytical
    simplifications on the original FTHA Otto model, with other minor improvements that are highlighted as they appear.


%------------------------------------------------------------------------------------------------------------------------------%
%                                                   Improved FTHA Otto Model                                                   %
%------------------------------------------------------------------------------------------------------------------------------%

\section{Improved FTHA Otto Model}



